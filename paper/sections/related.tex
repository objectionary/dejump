% SPDX-FileCopyrightText: Copyright (c) 2022 Mikhail Lipanin
% SPDX-License-Identifier: MIT

The problem was firstly formulated by \citet{dijkstra1968letters}: he affirmed that the use of GOTO has disastrous effect and proposed to remove it from all programming languages.
Even though a few years later \citet{knuth1974structured} showed a manner for structuring programs with the GOTO statement and argued that it could be a powerful tool if it is well used, researchers got interested in finding a way for restructuring programs and eliminating GOTO statements \citep{arsac1977construction}.

To solve the problem of eliminating Jump-like statements in other programming languages, the following works were published:
\Citet{williams1985restructuring} created a tool for restructuring Pascal code;
\Citet{jonsson1989next} was one of the first who made a summary of so called GOTO-patches, which are design techniques for eliminating GOTO statements;
\Citet{erosa1994taming} suggested an algorithm for automatic elimination of GOTO statements in~C;
\Citet{zegour1994new} explained how GOTO elimination may work in any language, as long as it has jump statements and control structures;
\Citet{morris1997goto} suggested to use regular expressions for replacing GOTO with IF-THEN-ELSE constructs;
\Citet{ramshaw1988eliminating} proposed an algorithm for Pascal programs;
\Citet{ganapathi2008shim} introduced optimization method for SHIM IR;
\Citet{ceccato2008goto} suggested how to eliminate GOTO during migration of legacy code to Java.

However, to our knowledge, there is still no method available, which enables automatic elimination of jump objects in programs written in such an object-flow languages.
