% SPDX-FileCopyrightText: Copyright (c) 2022 Mikhail Lipanin
% SPDX-License-Identifier: MIT

Let's consider the following example of C++ code, which uses goto:
\begin{ffcode}
int x = 0;|$\label{ln:cpp0}$|
int func() {
l1:
  x = x + 1;
  if (x < 10)
    goto l1;
  else
    cout << "X >= 10!\n";
  return x;
}|$\label{ln:cpp1}$|
\end{ffcode}
Applying the method described by ~\citet{erosa1994taming}, we get the following code:
\begin{ffcode}
int goto_l1 = 0;|$\label{ln:tcpp0}$|
int x = 0;
int func() {
  do {
    goto_l1 = 0;
    x = x + 1;
    if (x < 10)
      goto_l1 = 1;
    else
      cout << "X >= 10!\n";
  } while (goto_l1);
  return x;
}|$\label{ln:tcpp1}$|
\end{ffcode}

Since we consider the problem of eliminating Jump-like statements on the example of a strictly object-oriented language EO, \lrefs{cpp0}{cpp1} will be represented as:
\begin{ffcode}
memory 0 > x
[] > func
  goto > @
    [g]
      seq > @
        x.write (x.plus 1)
        if.
          x.lt 10
          g.backward
          QQ.io.stdout "X >= 10!\n"
        g.forward x
\end{ffcode}
After the transformations that we propose, the code above is transformed into:
\begin{ffcode}
memory 0 > fl_1
memory -1 > fl_2
memory 0 > x
[] > func
  seq > @
    fl_1.write 0
    fl_2.write -1
    while. > rem!
      or.
        fl_1.eq 0
        fl_2.eq -1
      [i]
        seq > @
          fl_1.write 1
          fl_2.write 0
          x.write (x.plus 1)
          if.
            x.lt 10
            fl_1.write 0
            QQ.io.stdout "X >= 10!\n"
          if.
            fl_1.eq 1
            fl_2.write 1
            TRUE
    if.
      fl_2.eq 1
      x
      rem
\end{ffcode}

Thus, the main purpose of our research is:
\begin{enumerate}
    \item To formulate inference rules for the transformations that we propose,
    \item proof that these 4 transformations are necessary and sufficient to solve the problem under consideration,
    \item suggest a program tool, which
    enables automatic elimination of jump objects in programs written in EO.
\end{enumerate}
